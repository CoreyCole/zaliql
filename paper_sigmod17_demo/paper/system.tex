
\begin{figure}
 \includegraphics[scale=0.25]{Figures/System-Overview.png}
 \caption{\GSQLB\'s Architecture}
  \label{fig:arch}
  \vspace{-3mm}
\end{figure}

\begin{figure}
\hspace*{.3cm}\includegraphics[scale=0.2]{figures/Scenario-Graph.png}
\caption{Confounding Influence}

\label{fig:cv}
\vspace{-0.3cm}
\end{figure} 

\section{System's Architecture}

The overall architecture of \GSQL\ is shown in Figure \ref{fig:arch}. 
The API consists of a set stored procedures support a wide range of method for performing causal inference.  There are written in PL/pgSQL and are optimized for PostgreSQL. \GSQL\
The API is packed as a an extension of PostgreSQL.
The functionality of the API is modeled after the MatchIt and CEM R libraries \cite{ho2005,iacus2009cem}.
The web GUI (see Figure \ref{fig:inteface}) is configured with the user's databases URL after they have installed the \GSQL\ Postgres extension.
It can be hosted remotely on a server or locally on your machine. \ignore{
The interface consists of three tabs : Setup, Matching Summary, and Analysis.
The {\em Setup} tab allows the user to specify their Treatments, Covariates for each treatment, and Outcomes.
Clicking on the {\it ADD TREATMENT}, {\it ADD COVARIATE}, and {\it ADD OUTCOME} buttons will trigger a dialog allowing
  further customization on the discretization and matching methods.
These dialogs will also allow the user to incorporate other columns (e.g. snow is defined as $iff Precipm> 0.3 and
Tempm< 0$) and input SQL for custom discretization methods.}
