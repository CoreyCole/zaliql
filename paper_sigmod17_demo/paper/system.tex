
\begin{figure}\center
 \includegraphics[scale=0.15]{Figures/System-Overview.png}
  \vspace{-3.3mm} \caption{\GSQLB architecture}

  \label{fig:arch}
  \vspace{-3mm}
\end{figure}


\begin{figure} \center
  \includegraphics[scale=0.18]{Figures/Matching-Flowchart.png}
  \vspace{-3mm}\caption{Causal analysis workflow}

\label{fig:flowchart}
\vspace{-0.3cm}
\end{figure}

\vspace{-.2cm}

\section{System Architecture}

The overall architecture of \GSQL\ is shown in Fig. \ref{fig:arch}.
% \dans{corey addressed - start by saying that data is stored in a Postgres system}
The API is a set of funtions that support a wide
range of methods for performing causal inference on data stored in PostgreSQL DBMS. %\footnote{\url{http://postgresql.org/}}
The API will be packaged as a PostgreSQL extension. %and will be released in
%PostgreSQL Extension Network.\footnote{\url{http://pgxn.org/}}
The functionality of the API is modeled after the MatchIt and CEM toolkits 
\cite{ho2005,iacus2009cem} and includes methods for drawing causal inference from relational data. \GSQL\ also includes a web GUI
(see Fig. \ref{sfig:demo-tutorial}) that can be hosted either locally or
remotely. %It can be configured with the user's databases URL after they have installed the \GSQL\ Postgres extension. The  GUI be hosted remotely on a server or locally on your machine. \ignore{
%The interface consists of three tabs : Setup, Matching Summary, and Analysis.
%}

\ignore{
\dans{We should add more details here: describe the main functions
  that the API supports; mention that they generate and optimize SQL
  queries.}}
