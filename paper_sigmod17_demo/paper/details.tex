\section{Demonstration Details}
\label{sec:dd}

The demonstration starts with weather and flight data.   

\ignore{
\begin{enumerate}
\item {\it Selecting a causal hypothesis:}
  the attendee selects a causal hypothesis to test by specifying a treatment (potential cause)
  and an outcome (potential effect) (cf, Example \ref{sfig:testaa}).  
\item {\it Matching method:} The user select a matching method to 
\item Enable attendees to drill into the details of how we generate these
results (i.e.\, the different steps of the pipeline).
\item Show what happens for different configurations of the Viska system:
e.g.\, different parameters to graph-generation algorithms, different bucketing
strategies, etc.
\end{enumerate}
}

\begin{enumerate}
  \item {\it Form a causal hypothesis:}
    e.g. snow causes flight delays
  \item {\it Setup database parameters:} (cf, Example \ref{sfig:testaa})
    \begin{enumerate}
      \item specify treatments (potential cause)
      \begin{enumerate}
        \item specify at least one treatment
        \item specify a matching method for each of those treatments (e.g. propensity score matching)
      \end{enumerate}
      \item for each treatment, specify a set of covariates (matching attributes)
      \item specify at least one outcome (potential effect)
    \end{enumerate}
  \item {\it View matching assesment:} (cf, Example \ref{sfig:testbb})
    \begin{enumerate}
      \item if any one treatment does not have sufficient results, one can re-run the procedure, for just that treatment,
        using a different matching method (i.e. Thunder in \ref{sfig:testbb})
    \end{enumerate}
  \item {\it Export for further analysis:} (cf, Example \ref{sfig:testcc})
    \begin{enumerate}
      \item export data for any treatment, or all treatments, to CSV file format
    \end{enumerate}
\end{enumerate}


\ignore{
\paragraph{Performance Data.}

Viska generates and analyzes performance data obtained by benchmarking
systems. This performance data consists of a set of records
each representing a single execution of an experiment. This record has
a set of ``input'' variables that we vary between experiments, such as
type of queries run, data set used, or configuration parameters. Viska
records these along with a set of ``observed'' variables: properties
of the execution (e.g., number of rows processed), system-level
metrics (e.g., CPU and disk utilization), and outcome variables (e.g.,
query execution time).

The Viska toolkit includes tools to generate and analyze this
performance data. We do not describe the tools for workload
generation, system deployment, and metric gathering here.
}
